% Concepts
% =========

\chapter{Solutions exercices}
\label{ch:solExer}

% Les exercices apparaisssnet dan sle syllabus sius la forme 
% \begin{myexercice} \label{exerc:test1}
% Enoncé de l'exercice
% \end{myexercice}
%
% La solution dans ce chapitre est rédigée sous la forme
% \solexercice{exerc:test1}
% Solution de l'exercice

%------------------------------------------------------
\section{Introduction}





%------------------------------------------------------
\section{Concepts}

 \solexercice{exerc:conceptfct}
% Solution de l'exercice



%------------------------------------------------------
\section{Resultats Fondamentaux}




%------------------------------------------------------
\section{Modèles de calculabilité}

% Solution exercice: réduire 1 + 2 (4.81)
\solexercice{exerc:reduire12}
Représentons d'abord A et 2 en expressions lambda.
$$\begin{array}{ll}
	1 \Leftrightarrow \lambda f \lambda c (fc)\\
	2 \Leftrightarrow \lambda f \lambda c (f(fc))\\
\end{array}$$
Par la règle de l'addition(\ref{prop:additionlambda}), nous trouvons
$$\begin{array}{ll}
	[\textcolor{red}{1}+\textcolor{blue}{2}] & \Leftrightarrow \lambda a \lambda b ( ( ( \textcolor{red}{\lambda f \lambda c (fc)} )a ) ( ( ( \textcolor{blue}{\lambda f \lambda c (f(fc))})a)b) ) \\
	& \Leftrightarrow \lambda a \lambda b ((\textcolor{red}{\lambda c (a c)}) ((\textcolor{blue}{\lambda c (a(a(c)))})b)) \\
	& \Leftrightarrow \lambda a \lambda b ((\textcolor{red}{\lambda c (a c)}) (\textcolor{blue}{(a(a(b)))}))\\
	& \Leftrightarrow \lambda a \lambda b (a(a(a(c))))
\end{array}$$
Ce qui est bien la représentation lambda de 3.
%------------------------------------------------------
\section{Analyse de la thèse de Church-Turing}




%------------------------------------------------------
\section{Complexité}




%------------------------------------------------------
\section{Classes de complexité}

