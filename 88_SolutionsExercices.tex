% Concepts
% =========

\chapter{Solutions exercices}
\label{ch:solExer}

% Les exercices apparaisssnet dan sle syllabus sius la forme 
% \begin{myexercice} \label{exerc:test1}
% Enoncé de l'exercice
% \end{myexercice}
%
% La solution dans ce chapitre est rédigée sous la forme
% \solexercice{exerc:test1}
% Solution de l'exercice

%------------------------------------------------------
\section{Introduction}





%------------------------------------------------------
\section{Concepts}

 \solexercice{exerc:ensembleEnumNumAlgebrique}
 Posons $E_k$ l'ensemble des nombres algébriques qui satisfont le polynôme:\\
   $a_0 + a_1x + a_2x^2 + ... + a_nx^n$. avec n<k et max($|c_j|$) < k.\\
 Il existe au plus $k^k$ polynômes de cette forme,
et chacun a au plus k racines.\\
Donc $E_k$ est un ensemble fini ayant au plus $k^{k+1}$ éléments. Ainsi nous listons les nombres algébriques $E_1, E_2, E_3, E_4$ et retirons les répétitions et obtenons une nouvelle ensemble qui est enumerable.
 
 
 \solexercice{exerc:conceptfct}
% Solution de l'exercice


\solexercice{exerc:ensembleNonEnumNumTranscendantaux}
Soit A l’ensemble des nombres algébriques et T l’ensemble des nombres transcendantaux.\\
Notons que l'ensemble $R = A \cup T$ et A est énumérable. Si T étais énumérables, alors R serait l’union de deux ensembles énumérables. Puisque R est non-énumerable, par conséquent, T est non-enumerable cas s'il etais R serait énumerable.

%------------------------------------------------------
\section{Resultats Fondamentaux}




%------------------------------------------------------
\section{Modèles de calculabilité}




%------------------------------------------------------
\section{Analyse de la thèse de Church-Turing}




%------------------------------------------------------
\section{Complexité}




%------------------------------------------------------
\section{Classes de complexité}

