% Complexité
% ============

\chapter{Complexité}
\label{sec:complexit_}
Lors de l'étude de la complexité, on ne va considérer que la borne supérieure
(notation big O). De plus, on ne va considérer que les fonctions totales et
donc la décision d'ensembles récursifs.
En effet, si la fonction n'est pas totale, l'algorithme peut boucler.
Donc on ne sait pas étudier l'efficacité.

\begin{mydef}[Complexité d'un problème] Complexité de l'algorithme le
	\textbf{plus efficace} résolvant ce problème.
\end{mydef}

\begin{mydef}[Problème pratiquement faisable]
	S’il existe un algorithme de complexité polynomiale qui résout ce
	problème, alors, celui-ci est pratiquement faisable.
\end{mydef}

\begin{mydef}[Problème intrinsèquement complexe]
	S’il n'existe pas d'algorithme de complexité polynomiale qui résout ce
	problème, alors, celui-ci est intrinsèquement complexe.
\end{mydef}

\begin{myrem}
	Quelle est la différence entre intrinsèquement complexe et pratiquement
	infaisable?
\end{myrem}

\section{Influence du modèle de calcul}
\label{sub:influence_du_mod_le_de_calcul}
Si un algorithme est de complexité polynomiale dans un modèle complet alors il sera
polynomial dans un autre modèle de calcul. Il y aura juste un facteur
polynomial entre les deux, car il existe un compilateur du premier modèle vers
le second qui a une complexité polynomiale.
% subsection influence_du_mod_le_de_calcul (end)

\section{Influence de la représentation des données}
\label{sub:influence_de_la_repr_sentation_des_donn_es}
Le choix de représentation de données induit une variation polynomiale du temps
d'exécution et de l'espace.

\begin{myrem}
	Certains problèmes sont intrinsèquement complexes juste pour certaines
	données (simplexe). Celles-ci sont souvent des cas particuliers et peu
	rencontrées en pratique.
\end{myrem}
% subsection influence_de_la_repr_sentation_des_donn_es (end)

% section complexit_ (end)


