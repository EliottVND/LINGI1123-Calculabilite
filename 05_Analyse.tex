% Analyse
% ========

\chapter{Analyse de la thèse de Church-Turing}
\label{sec:analyse_de_la_th_se_de_church_turing}
Dans ce chapitre, on va principalement voir ce qu'est un bon formalisme de calculabilité.

% section fondement de la thèse (start)
\section{Fondement de la thèse}
\label{sub:fondement_de_la_th_se}
\begin{mythese}[Thèse de Church-Turing]
La forme originale ne contient que 2 parties :
\begin{enumerate}
	\item Toute fonction calculable par une MT est effectivement calculable ;
	\item Toute fonction effectivement calculable est calculable par une MT.
\end{enumerate}
\end{mythese}

La partie 1 est démontrée (c'est évident puisqu'on peut simuler les machines de Turing).\\

La partie 2 est supposée vraie : on ne peut pas la démontrer formellement car ``effectivement calculable'' n'est pas bien défini. 
On la suppose vraie, car on a des évidences qui supportent la deuxième partie. Ces évidences sont de deux types :
\begin{itemize}
\item Evidences heuristiques ;
\item Equivalence des divers formalismes.
\end{itemize}
\subsection{Evidences heuristiques}
Quelques évidences de cette première gamme de preuves sont montrées :
\begin{itemize}
\item Beaucoup de tentatives de trouver une fonction effectivement calculable qui ne soit pas calculable par une machine de Turing. Aucune de ces tentatives ne fut un succès. Echec sur toute la ligne.
\item Toutes fonctions particulières s'avérant effectivement calculables, peu importe le formalisme utilisé, ont été montrées être calculables par une machine de Turing. Notons que beaucoup d'essais ont été réalisés couvrant une très large classe de problèmes.
\item Des méthodes ont été développées pour montrer qu'une fonction effectivement calculable peut être calculée par une machine de Turing.
\end{itemize}

\subsection{Equivalence des divers formalismes}
\begin{itemize}
\item Différents formalismes, très différents, basés sur des principes distincts, ont été proposés pour définir la calculabilité. Ceux-ci possèdent les mêmes évidences heuristiques que pour les machines de Turing. 
\item On a montré l'équivalence  entre tous les formalismes de calculabilité crées à ce jour. On a montré que toute machine constructible par la mécanique newtonienne ne calcule que des fonctions calculables.

\end{itemize}


%On la suppose vraie, car on a des évidences heuristiques (il y a eu beaucoup d'essais pour trouver une fonction qui ne respectait pas la thèse) et qu'on a montré l'équivalence entre tous les formalismes créés à ce jour (on a montré que toute machine constructible par la mécanique de Newton ne calcule que des fonctions calculables).
% section fondement de la thèse (end)

% section formalisme de la calculabilité (start)
\section{Formalismes de calculabilité}
\label{sub:formalismes_de_la_calculabilit_}

On se pose la question de ce qui fait un bon formalisme pour la
calculabilité. Il faut un formalisme qui vérifie les fondements de la
thèse énoncée précédemment. Plus précisément, on va étudier les caractéristiques (les propriétés)
nécessaires et suffisantes pour avoir un bon modèle de calculabilité.

\paragraph{} Dans ce chapitre, on considére un nouveau formalisme de calculabilité D.

\begin{samepage}
\paragraph{Caractéristiques}
\begin{description}
	\item[SD] Soundness\footnote{Cohérence} des descriptions
	\item[CD] Complétude des descriptions
	\item[SA] Soundness algorithmique
	\item[CA] Complétude algorithmique
	\item[U] Description universelle\nopagebreak
	\item[S] Propriété S-m-n affaiblie
\end{description}
\end{samepage}

\begin{myrem}
	\textbf{D} correspond à ``description'', c'est-à-dire une description de fonction.\\
	\textbf{A} correspond à ``algorithme'', c'est-à-dire à une description exécutable.
\end{myrem}

\begin{myrem}
	\textbf{Soundness} signifie que, si on a une description dans notre modèle D alors celle-ci est cohérente, et correspond bien à une fonction calculable.\\
	\textbf{Complétude} signifie que notre modèle est complet, qu'il n'existe pas de fonction calculable qui ne le soit pas dans notre modèle.
\end{myrem}

\paragraph{Caractéristiques détaillées}
\begin{description}
	\item[SD]  Toute fonction D-calculable est calculable (première partie de la thèse de Church-Turing). Si il existe une description D $\in \textit{D}$ calculant une fonction f, alors cette fonction est calculable.
	\item[CD]  Toute fonction calculable est D-calculable (deuxième partie de la thèse de Church-Turing). Si une fonction f est calculable, alors il existe une description D $\in \textit{D}$ calculant cette fonction f.
	\item[SA]  L'interpréteur de D est calculable. Toute description D $\in \textit{D}$ peut être effectivement exécutée sur des données. Le formalisme D est algorithmique. Cette caractéristique assure que D décrit bien des procédés effectifs.
	\item[CA]  Il existe un compilateur qui étant donné un programme $p$ dans un formalisme respectant SA produit une description de programme $d \in D$ tel que $p$ et $d$ calculent la même fonction (cette propriété permet d'assurer l'équivalence des formalismes).
	\item[U]  L'interpréteur de D est D-calculable. Le formalisme doit permettre de décrire son propre interpréteur sinon on sait par Hoare Allison que ce n'est pas un formalisme complet.
	\item[S] Pour tout programme $d \in$ D à deux arguments, il existe un transformateur de 
  programme $S$ calculable, qui 
  recevant comme entrée le programme $d$ et une valeur 
  $x$ fournit comme résultat un programme $d'=S(d,x)$ tel que $d'(y)$ calcule 
  la même fonction que $d(x,y)$ pour cette valeur de $x$ :
$$\forall d\in D\ \exists S\in D \colon \varphi_d(x,y)=\varphi_{S(d,x)}(y).$$
\end{description}
Un bon formalisme de calculabilité possède toutes ces propriétés. En pratique, il suffit d'en montrer certaines, car certaines propriétés en entraînent d'autres.

\begin{myprop}
S et U $\Rightarrow$ S-m-n.
\end{myprop}

\begin{myprop}
SA $\Rightarrow$ SD.
    
\begin{proof}
Si l'interpréteur est calculable, en particulier l'interpréteur restreint à un programme est calculable, donc toute fonction D-calculable est calculable.
\end{proof}
\end{myprop}

\begin{myprop}
CA $\Rightarrow$ CD.
    
\begin{proof}
Dans la propriété CA, prenons Java comme formalisme respectant SA (on sait qu'être Java-calculable est équivalent à être calculable). Par CA, tout programme de Java est équivalent à un programme dans D (ils calculent la même fonction), donc toute fonction Java-calculable (c'est-à-dire calculable) est D-calculable.
\end{proof}
\end{myprop}

\begin{myprop}
SD et U $\Rightarrow$ SA.
    
\begin{proof}
Par U, l'interpréteur est D-calculable. Par SD, il est calculable, donc on a SA.
\end{proof}
\end{myprop}

\begin{myprop}
CD et S $\Rightarrow$ CA.
    
\begin{proof}
Considérons n'importe quel autre formalisme $P$ qui a la propriété SA (ce qui signifie que son interpréteur $\interpret_P(n,x)$ est calculable). Donc par la propriété CD, il existe un programme $d\in D$ tel que $\varphi_d(n,x)=\varphi_{\interpret_P}(n,x)$. Or par la propriété S, il existe T un transformateur de programme tel que : $\varphi_d(n,x)=\varphi_{T(n)}(x)$. On a donc :
$$\varphi_n(x) = \varphi_{\interpret_P}(n,x) = \varphi_d(n,x)=\varphi_{T(n)}(x).$$
On peut donc voir T comme un programme qui compile/transforme une description $p\in P$ en une description $d\in D$. Ce qui implique qu'on a bien la propriété CA.
\end{proof}
\end{myprop}

Un bon formalisme doit donc posséder soit SA et CA ou soit SD, CD, U et S ou soit SA, CD et S ou encore CA, SD et U, car
\begin{itemize}
	\item SA et CA $\iff$ SD, CD, U et S
	\item SA, CD et S $\iff$ SD, CA et U
\end{itemize}
BLOOP et les fonctions primitives récursives sont des ``mauvais'' formalismes car ils ont seulement les propriétés SD, SA et S.
% section formalismes de la calculabilité (end)

% section Techniques de preuve (start)
\section{Techniques de preuves}
\label{sub:techniques_de_preuve}

Pour prouver qu'un problème ou une fonction est non calculable, on peut utiliser :
\begin{itemize}
	\item Le théorème de Rice.
	\item La démonstration directe de la non-calculabilité par 
  diagonalisation ou par preuve par l'absurde. 
  \textbf{A priori le plus dur!} C'est plus pratique de réutiliser les problèmes pour lesquels on a déjà montré la non-calculabilité.
	\item La méthode de réduction.
\end{itemize}
% section Techniques de preuve (end)

% section Aspect non couverts par la calculabilité (start)
\section{Aspects non couverts par la calculabilité}
\label{sub:aspects_non_couvert_par_la_calculabilit_}
La calculabilité se limite au calcul de fonctions. Or certains problèmes de la vie de tous les jours ne correspondent pas à une fonction.

\begin{myexem}
Système d'exploitation.
\end{myexem}

\begin{myexem}
Système de réservation aérienne.
\end{myexem}

\begin{myexem}
Certains problèmes utilisent des caractéristiques de l'environnement comme des données en provenance de sondes extérieures.
\end{myexem}
% section aspects non couverts par la calculabilité (end)

% section Au-delà de la calculabilité (start)
\section{Au-delà de la calculabilité}
\label{sub:au_del_de_la_calculabilit_}
En définissant la calculabilité, on met en évidence des questions qui vont au-delà de son champ d'application :
\begin{itemize}
	\item Est-il possible d'imaginer un modèle plus puissant que les modèles qu'on a aujourd'hui ?
	\item Voir même, est-ce que nos modèles de calculabilité sont limités par l'intelligence humaine ?
	\item Inversement, est-ce que les humains sont ``plus puissants'' que les machines dans le sens où ils pourraient calculer des fonctions non calculables ?
	\item Plus précisément dans le cas des machines de Turing : peut-on simuler tout processus mental par une machine de Turing ?
\end{itemize}
 

\subsection{Calculabilité et intelligence humaine : les thèses CT}
Les thèses CT sont des approches à la question de l'équivalence entre les fonctions calculables par une machine et les fonctions calculables par un être humain.

\begin{mydef}[H-calculable]
Une fonction est H-calculable si un être humain est capable de calculer cette fonction.
\end{mydef}

Les deux thèses suivantes sont deux extrêmes :
\begin{mythese}[version ``réductionniste'']
Si une fonction est H-calculable, alors elle est T-calculable car :
\begin{enumerate}
	\item le comportement des éléments constitutifs d'un être vivant peut être simulé par une machine de Turing;
	\item tous les processus cérébraux dérivent d'un substrat calculable.
\end{enumerate}
\end{mythese}

\begin{mythese}[version ``anti-réductionniste'']
Certains types d'opérations effectuées par le cerveau, mais pas la majorité d'entre elles (et certainement pas les plus intéressantes), peuvent être exécutées de façon approximative par les ordinateurs.\\
Autrement dit, certains aspects seront toujours hors de portée des ordinateurs :
\begin{center}
H-calculable $\neq$ T-calculable
\end{center}
\end{mythese}

La thèse suivante cherche à limiter le problème aux fonctions ``difficiles à définir'' :
\begin{mythese}[version ``procédés publics'']
Si une fonction est H-calculable et que l'être humain calculant cette fonction est capable de décrire (à l'aide du langage) à un autre être humain sa méthode de calcul de telle sorte que cet autre être humain soit capable de calculer cette fonction, alors la fonction est T-calculable.
\end{mythese}

\begin{myrem}
Cette thèse n'exclut pas l'existence de fonctions H-calculables, mais non T-calculables \dots
\end{myrem}

Aucune réponse scientifique à cette question ne peut être donnée, seules différentes positions sont possibles :
\begin{itemize}
	\item \textbf{``Croyant''} : H-calculable $=$ T-calculable (exemple : réductionniste).
	\item \textbf{``Athée''} : H-calculable $\neq$ T-calculable (exemple : anti-réductionniste).
	\item \textbf{``Agnostique''} : pas intéressé par la question.
	\item \textbf{``Iconoclaste''} : la question est mal posée, elle n'a pas de sens.
\end{itemize}

\paragraph{Lien avec l'IA}
Deux approches possibles pour l'intelligence artificielle :
\begin{enumerate}
	\item \textbf{IA forte} : simulation du cerveau humain à l'aide d'un ordinateur. En copiant le fonctionnement, on espère ainsi obtenir des résultats similaires.
	\item \textbf{IA faible} : programmation de méthodes et techniques de raisonnement en exploitant les caractéristiques propres des ordinateurs.
\end{enumerate}
La première approche (IA forte) s'apparente à la version ``réductionniste'' de la thèse de Church-Turing.


\paragraph{Approche par la puissance de calcul}
On peut tenter de déterminer ``l'intelligence'' en observant simplement la puissance de calcul :
\begin{itemize}
	\item Puissance du cerveau humain estimée : entre $10^{13}$ et $10^{19}$ instructions par seconde.
	\item Ordinateurs les plus puissants :
	\begin{itemize}
	\item IBM ASCI White, Loas Alamos, USA : $13.9$ TFLOPS ;
	\item Earth Simulator, Japan : $35,8$ TFLOPS (5000 processors).
\end{itemize}
Puisque $1$ FLO est environ $2$ à $10$ instructions, cela fait une puissance d'environ $10^{14}$ instructions par seconde.
\end{itemize}
Conclusion : les ordinateurs les plus puissants approchent la puissance de calcul du cerveau humain.

\subsection{Les ordinateurs peuvent-ils penser ?}
\paragraph{Le test de Turing (1950)}
On compare deux tests similaires.\\

\emph{Test de base} : 3 personnes, un homme (\textbf{A}), une femme (\textbf{B}) et un interrogateur communiquent exclusivement via machines à écrire. L'interrogateur posent des questions pour deviner qui est l'homme et qui est la femme. L'homme (\textbf{A}) doit faire perdre l'interrogateur, la femme (\textbf{B}) le faire gagner.\\

\emph{Test de Turing} : une machine (\textbf{A}), un être humain (\textbf{B}) et un interrogateur communiquent exclusivement via machines à écrire. L'interrogateur posent des questions pour deviner qui est la machine et qui est l'être humain. La machine (\textbf{A}) doit faire perdre l'interrogateur, l'être humain (\textbf{B}) le faire gagner.\\

Si l'interrogateur se trompe autant de fois dans le premier test que dans le second, on en conclut que les machines peuvent penser.

\paragraph{Raisonnement de Turing}
Turing propose le raisonnement suivant pour répondre à la question ``les machines pensent-elles ?'' :
\begin{itemize}
	\item \textbf{Hypothèse :} les machines pensent ;
	\item si on ne peut réfuter l'hypothèse alors celle-ci est vraie ;
	\item envisager toutes les objections (théologique, émotions, mathématiques...) et toutes les réfuter.
\end{itemize}
% section au_del_de_la_calculabilit_ (end)
% chapter Analyse de la thèse de Church-Turing (end)
